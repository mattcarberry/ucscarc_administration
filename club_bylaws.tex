\documentclass{report}
\usepackage[normalem]{ulem}

\begin{document}

\title{UC Santa Cruz Amateur Radio Club Bylaws \\ \small{Ratified March 12, 2014}}
\author{}
\date{}
\maketitle

\tableofcontents

\pagebreak

\chapter{Secretary}
\begin{enumerate}
\item It shall be the duty of the Secretary to keep the constitution and Bylaws of the club and have the same with him/her at every meeting. He/she shall note all amendments, changes and additions to the constitution and shall permit it to be consulted by members upon request.

\item The secretary shall be responsible for reviewing content on the club's website and shall have special privileges to edit and delete all content to prudently ensure accuracy and continuity. 

\end{enumerate}

\chapter{Membership}
\begin{enumerate}

\item Full membership is open to UCSC student licensed Radio Amateurs.

\item Full membership includes all club privileges as well as rights to hold a club office and to vote for club officers.

\item Associate membership is open to those actively engaged in a class leading to an Amateur Radio license and all other interested persons.

\item Associate membership includes all club privileges except for the right to hold office and vote for club officers.

\item Advisory membership is open to UCSC ARC alumni, UCSC staff and faculty, or a member of good standing with the ARRL that also has an interest in the development of UCSC ARC.

\item Advisory membership includes all club privileges except for university lab and associated access. Advisory members will vote to elect three representatives holding active amateur radio licenses to constitute the Advisory Committee (AC). The AC is expected to be the general voice of all Advisory members. The AC shall have one vote at the officer level to be determined by a majority of its members.

\item Applications for membership shall be submitted at regular meetings.
\end{enumerate}


\chapter{Meetings}
\begin{enumerate}
\item Regular meetings shall be determined for each academic quarter to accommodate member's academic schedules at the first pre-scheduled meeting of any quarter. Scheduling the first meeting for a successive quarter shall be a standing agenda item at the last meeting of the previous quarter. When warranted, special meetings may be called by the President. Notices shall be sent to members concerning special meetings and the business to be transacted. Only such business as designated shall be transacted. Such notices shall be sent so that they arrive not less than 48 hours before the meeting and be posted on the Club's website.
\end{enumerate}

\chapter{Elections}
\begin{enumerate}
\item Officer elections will be held annually at the regular March meeting.

\item Officers will assume their elected office April 1 of the following year.
\end{enumerate}

\section{Nomination Committee}
\begin{enumerate}
\item The Vice President will appoint a nomination committee at the regular February meeting.

\item The committee will be charged with finding qualified candidates to run for office.

\item The committee will present a ballot to the president at the regular meeting one month prior to elections at which time the President will entertain a motion for nominations and to accept the ballot.

\item The final ballot will be published and made available to all club members on the club website.

\item The Vice President will dissolve the nomination committee after the annual elections.
\end{enumerate}

\section{Voting}
\begin{enumerate}
\item Voting will be done using paper ballots.

\item Immediately after voting, the ballots will be counted openly in front of the membership.

\item Any candidate may request a recount of the ballots.

\item In the case of an unopposed slate the President may entertain a motion for the Advisory Committee to cast a single ballot.

\item In the case of more than two candidates running for a single office, there will be a run-off between the top two remaining candidates.  

\item Removal of Advisory Committee members shall be left to their discretion or through a majority vote of the officers.
\end{enumerate}

\chapter{Interference}
\begin{enumerate}
\item The responsibilities of handling interference complaints with other services will fall to the Advisory Committee.
\end{enumerate}

\chapter{Amateur Radio Emergency Service (ARES)}
\begin{enumerate}
\item ARES® is a program of the American Radio Relay League -- our club will abide by the Rules and Regulations of the ARRL’s Field Organization as they may be amended from time to time, and by ARRL policies, rules, and guidelines contained in ARRL publications.
\end{enumerate}

\chapter{Membership Communication}
\begin{enumerate}
\item Official club activities shall be communicated to all members by the club's website.  This shall minimally include: Meeting times; Minutes of meetings posted within 24 hours by the club secretary following any scheduled or special meeting; events and activities as determined by the club.
\end{enumerate}

\end{document}